\documentclass[../main]{subfiles}
\graphicspath{{\string~/AQUOS/Default_Folder/TIU/lectures/05_R/tex/fig/}}
\begin{document}

\tcbstartrecording\relax

\section{テキストデータの出力}

\Definition{手順}
{
  \cs{write.csv}関数を使用して,オブジェクトデータを
  ファイルにCSVファイル形式*で書き込む.
  * CSV:カンマ区切
}

\begin{ConsoleR}
  d0 <- data.frame(name = c('panda', 'lion'), 
                    age = c(5, 7))
  write.csv(d0, file = 'd0.csv')
\end{ConsoleR}

\cs{quote = F} オプションをつけると文字列引用符「"」を削除できる.
\cs{write.csv(d0, file = 'd0.csv', quote = F)}

\begin{exercise}
  データフレームを作成し,ファイルに出力してください.
\tcblower
\end{exercise}

\section{テキストデータの入力}

\Definition{手順}
{
  \cs{read.csv}関数を使用して,CSVファイルを読み込みオブジェクトに格納する.
}

\begin{ConsoleR}
  d1 <- read.csv(file = 'd0.csv')
  str(d1)
\end{ConsoleR}

\cs{stringsAsFactors = F} オプションをつけると文字列の自動因子化を抑制します.
\cs{read.csv(file = 'd0.csv', stringsAsFactors = F)}

\begin{exercise}
  CSVファイルを読み込みオブジェクトに格納してください.
\tcblower
\end{exercise}

\section{\Excel データの入力}

\Definition{手順}
{
  \cs{excel.link}パッケージを利用し,\R とリンクさせる.\\
  パッケージの利用コマンド:\cs{library(excel.link)}
}

\begin{ConsoleR}
  library(excel.link)
  xl.workbook.open('test.xlsx') # Excelを開く
  d <- data.frame(x=3:1, y=-1:1)
  xl['Sheet1!A1'] <- d # Sheet1のA1を起点としてデータを書き込み
  xl['Sheet1!B2'] -> x # Sheet1のB2からデータを読み込み
  xl.workbook.save('test.xlsx') # Excelを保存
\end{ConsoleR}

\cs{xlr}: 行名付き,\cs{xlc}:列名付き, \cs{xlrc}:行列名付き入出力

\begin{exercise}
  パッケージのヘルプにあるサンプルコードを用いて\Excel を操作してください.
\tcblower
\end{exercise}


%=================================================================
\tcbstoprecording
\section{解答}
\tcbinputrecords

\end{document}
