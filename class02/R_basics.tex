\documentclass[../main]{subfiles}
\graphicspath{{\string~/AQUOS/Default_Folder/TIU/lectures/05_R/tex/fig/}}
\begin{document}

\tcbstartrecording\relax

\section{スカラの作成}

\Definition{手順}
{
  オブジェクト名の後に,\\
  代入(付置)記号「\cs{<-}」と値を入力する.
}
・「\cs{<-}」の代わりに「\cs{=}」も使用できる (若干意味が異なる).\\
・オブジェクト名は,大文字と小文字は区別される.
  %・ls()で作成したオブジェクトの名前を表示できる.\\
  %・rm('x')でオブジェクトxを削除できる. rm(list=ls())で全消去.\\



\begin{ConsoleR}
> x <- 1
> kw.pv <- 3.1
\end{ConsoleR}

\begin{ConsoleR}
> ls()            # オブジェクト名表示
> rm(list = ls()) # 全消去
\end{ConsoleR}

\R ではオブジェクト名称の大文字と小文字は区別される.
漢字名称も使用可能だが,通常はローマ字小文字で,
判別可能な略語を用い,ofの意味で「\cs{.}」や「\cs{\_}」
を利用し単語を結合すると分かりやすくなる.\\[5mm]
【例】\cs{lat.jp} (日本の緯度),\cs{kw.pv}(PV発電量)

\begin{exercise}
  「Alt~+~-」で代入記号(\cs{<-})を20回入力してください.\\
  オブジェクト名を表示,オブジェクトを削除してください.
  \tcblower
\end{exercise}

\section{ベクトルの作成}

\Definition{手順}
{
  \begin{description}
    \item[【方法1】] 結合関数「\cs{c}」を用いて作成
    \item[【方法2】] 等差数列作成記号「\cs{:}」を用いて作成
    \item[【方法3】] 等差数列作成関数「\cs{seq}」を用いて作成
  \end{description}
}

「\cs{rep}」関数で同一値ベクトル作成も可能\\
\cs{rep(NA, 3)} $\rightarrow$ NA~NA~NA\\
「\cs{?}関数名」をコンソールに入力するとヘルプが表示される.

\begin{ConsoleR}
> v <- c(1, 6, 3)
[1] 1 6 3

> v <- 1:3
[1] 1 2 3

> v <- seq(1, 6, 2)
[1] 1 3 5
\end{ConsoleR}

\begin{exercise}
  次のベクトルを作成してください.\\
  「3 2 1」,「3 6 9」,「4 2 0」,「1.5 2.5 3.5」,「1 2 3 1 2 3」
  \tcblower
\end{exercise}

\section{行列の作成}

\Definition{手順}
{
  行列作成関数「\cs{matrix}」を用いて作成.\\
  行数:\cs{nrow},列数:\cs{ncol}
}

\begin{ConsoleR}
> m <- matrix(1:4, nrow=2, ncol=2)
> m
     [,1] [,2]
[1,]    1    3
[2,]    2    4

> m <- matrix(1:4, nrow=2, ncol=2, byrow=T) # 行方向代入
> m
     [,1] [,2]
[1,]    1    2
[2,]    3    4
\end{ConsoleR}

\begin{exercise}
  次の行列を作成してください.
  \vspace{-9mm}
  \[
    \hspace{40mm}
    \begin{bmatrix}
      -2 & 0 & 2\\
        4 & 6 & 8
    \end{bmatrix}
    ,
    \begin{bmatrix}
      NA & NA\\
      NA & NA
    \end{bmatrix}
  \]
  \tcblower
  \begin{ConsoleR}
Ex.          [,1] [,2]
      [1,]   1    3
      [2,]   2    4
  \end{ConsoleR}
\end{exercise}

\section{オブジェクトの画面表示}

\Definition{手順}
{
  【方法1】コンソールに表示させたいオブジェクト名を入力する.\\
  【方法2】ソースコード画面でオブジェクトを選択してCtrl+Enterを押す.
}

\begin{ConsoleR}
> v[3]
[1] 3
> m[2, ]
[1] 2 4
> (x <- c(1, 2))
[1] 1 2
\end{ConsoleR}

【オブジェクト要素の\R 表記】\\
~~ベクトルvの要素$i$: \cs{v[i]}\\
~~行列mの$i$行: \cs{m[i, ]},$j$列: \cs{m[, j]},要素$(i, j)$: \cs{m[i, j]}

\begin{exercise}
  スカラ,ベクトル,行列の値を表示させてください.
  \tcblower
\end{exercise}

\section{データフレームの作成}

\Definition{手順}
{
  テーブル作成関数「\cs{data.frame}」を用いて作成する.
}

\begin{ConsoleR}
> d <- data.frame(name = c('panda', 'lion'),
                  age  = c(5, 7),
                  male = c(T, F))
> d
   name age  male
1 panda   5  TRUE
2  lion   7 FALSE
\end{ConsoleR}

\begin{exercise}
  漢字,数値,論理値のカラムを持つ,
  データフレームを作成してください(内容自由).
  \tcblower
\end{exercise}

\subsection{データフレームの操作1}

\Definition{手順}
{
  アクセスしたいコラム(列)やレコード(行)のインデック番号を入力する.
  負の番号を入れると,そのコラムが除かれる.
}

\begin{ConsoleR}
> d[, 1]
[1] "panda" "lion"
> d[, c(1, 3)]
   name  male
1 panda  TRUE
2  lion FALSE
> d[, -1]
  age  male
1   5  TRUE
2   7 FALSE
\end{ConsoleR}

\begin{exercise}
  レコード(行)にもインデックス番号でアクセスし,値の表示や代入を行ってください.
  \tcblower
\end{exercise}

\subsection{データフレームの操作2}

\Definition{手順}
{
  オブジェクト名のあとにアクセスしたいコラム(列)名を\cs{\$}で結びつける.
  または,コラム(列)名をリテラルで囲み記入する.
  データフレーム\cs{d}のカラム: \cs{d\$カラム名} or \cs{d[, 'カラム名']}
}

\begin{ConsoleR}
  > d@\$@age
  [1] 5 7

  > d[, c('name', 'age')]
     name age
  1 panda   5
  2  lion   7
\end{ConsoleR}

\begin{exercise}
  レコード(行)にもレコード名でアクセスし,値の表示や代入を行ってください.
  \tcblower
\end{exercise}

カラム名,レコード名は\cs{rownames(d)}, \cs{colnames(d)}でアクセス可能

\subsection{データフレームの操作3}

\Definition{手順}
{
  アクセスしたいコラム(列)やレコード(行)に論理値を入力する.
  \cs{T}(\cs{TRUE})の論理値箇所のデータが抽出される.
}

\begin{ConsoleR}
> d[, c(T, F, T)]
    name  male
1 panda  TRUE
2  lion FALSE

> d[d@\$@age > 6, ]
name age  male
2 lion 7 FALSE
\end{ConsoleR}

\begin{exercise}
  レコード(行)に論理値ベクトルでアクセスし,値の表示や代入を行ってください.
  \tcblower
\end{exercise}

\section{四則演算}

\Definition{手順}
{
  和「\cs{+}」,減「\cs{-}」,積「\cs{*}」,除「\cs{/}」,乗「~~\cs{\^} 」,
  剰余「\cs{\%\%}」,剰商「\cs{\%/\%}」の算法記号を使って演算する.
  要素ごとの演算となる.
}
この他,行列演算用の積「\cs{\%*\%}」,転置「\cs{t()}」,逆行列「\cs{solve()}」などもある.

\begin{ConsoleR}
> x <- 1:3; y <- 1:3
> x + y
[1] 2 4 6

> x <- 1:3; y <- 1:3
> x * y
[1] 1 4 9

> x <- 9; y <- 2
> x %/% y
[1] 4
\end{ConsoleR}

\begin{exercise}
  上記,すべての演算記号を使って計算してください(内容自由).
  \tcblower
\end{exercise}

\section{組込関数1}

\Definition{手順}
{
  平均値「\cs{mean}」,中央値「\cs{median}」,最大「\cs{max}」,最小「\cs{min}」\\
  範囲「\cs{range}」,平方根「\cs{sqrt}」,絶対値「\cs{abs}」,丸め「\cs{round}」など
}

\begin{ConsoleR}
> x <- 1:3
> mean(x)
[1] 2

> x <- 1:3
> range(x)
[1] 1 3

> x <- 3.14
> round(x, 1)
[1] 3.1
\end{ConsoleR}

\begin{exercise}
  上記,すべての組み込み関数を使って,計算してください(内容自由).
  また,その他の関数,定数(\cs{pi})は,どのようなものがあるかインターネットで検索してください.
  \tcblower
\end{exercise}

\section{自作関数1}

\Definition{手順}
{
  \cs{関数オブジェクト名 <- function (引数1,引数2,...) 関数式}\\
  の形式で関数を作成する.引数は値渡しとなる.
}

\begin{ConsoleR}
> get.mbe <- function(yhat, y) mean(yhat - y)
> mbe <- get.mbe(yhat = 1:3, y = 4:6)
[1] -3
\end{ConsoleR}

\begin{exercise}
  RMSE(平均2乗誤差平方)を求める関数を作成してください.
  \vspace{-2mm}
  \[
  \mathrm{RMSE} = \sqrt{\frac{1}{N} \sum_{i = 1}^{N} (\hat{y}_i-y_i)^2}
  \hspace{5mm} \mbox{平均:\cs{mean()},平方根:\cs{sqrt()}}
  \]
  \tcblower
\end{exercise}

\section{自作関数2}

\Definition{手順}
{
  複数行の関数を作成するときは,関数のスコープを示す\cs{{}}(中括弧)や
  変数名を指定して出力する\cs{return}関数を用いる.
}

\begin{ConsoleR}
> f <- function(x) {
      y <- 1 + x + x^2
      z <- log(y)
      return(z)
      }

> f(2)
[1] 1.94591
\end{ConsoleR}

\cs{return}関数では,最後のオブジェクトを返す場合は記述しなくてもよい.

\begin{exercise}
  複数行の関数を作成してください.
  \tcblower
\end{exercise}

%=================================================================
\tcbstoprecording
\section{解答}
\tcbinputrecords

\end{document}
