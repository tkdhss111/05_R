\def\MyCourse{データサイエンスコース}
\def\MySubject{R入門}
\def\MySemester{春学期}

\newcommand{\R}{\textbf{R}}
\newcommand{\RStudio}{\textbf{RStudio}}
\newcommand{\Excel}{\textbf{Excel}}
\newcommand{\cs}[1]{\textcolor{blue}{\texttt{#1}}} % Console prompt >


\subsection{テキストデータの出力}

\myffr

  \mybfr{手順}
    write.csv関数を使用して,オブジェクトデータを
    ファイルにCSVファイル形式*で書き込む.
    * CSV:カンマ区切
  \mybto
  

  \myefr{コンソール}
    \cs{ d0 <- data.frame(name = c('panda', 'lion'), age = c(5, 7)) }\\
    \cs{ write.csv(d0, file = 'd0.csv') }
  \myeto

  quote = Fオプションをつけると文字列引用符「"」を削除できる.
  write.csv(d0, file = 'd0.csv', quote = F)

  \mybfr{演習}
    データフレームを作成し,ファイルに出力してください.
  \mybto

\end{frame}

\subsection{テキストデータの入力}

\myffr

  \mybfr{手順}
    read.csv関数を使用して,CSVファイルを読み込み
    オブジェクトに格納する.
  \mybto

  \myefr{コンソール}
    \cs{d1 <- read.csv(file = 'd0.csv')}\\
    \cs{str(d1)}
  \myeto

  stringsAsFactors = Fオプションをつけると文字列の自動因子化を抑制します.
  read.csv(file = 'd0.csv', stringsAsFactors = F)

  \mybfr{演習}
    CSVファイルを読み込み,オブジェクトに格納してください.
  \mybto

\end{frame}

\subsection{\Excel データの入力}

\myffr

  \mybfr{手順}
    excel.linkパッケージを利用し,\R とリンクさせる.\\
    パッケージの利用コマンド: library(excel.link)
  \mybto

  \myefr{コンソール}
    \cs{library(excel.link)}\\
    \cs{xl.workbook.open('test.xlsx')}\hfill \mycheck{Excelを開く}\\
    \cs{d <- data.frame(x=3:1, y=-1:1)}\\
    \cs{xl['Sheet1!A1'] <- d}\hfill \mycheck{Sheet1のA1を起点としてデータを書き込み}\\
    \cs{xl['Sheet1!B2'] -> x}\hfill \mycheck{Sheet1のB2からデータを読み込み}\\
    \cs{xl.workbook.save('test.xlsx')}\hfill \mycheck{Excelを保存}
  \myeto

  \mycheck{xlr: 行名付き,xlc:列名付き, xlrc:行列名付き入出力}

  \mybfr{演習}
    パッケージのヘルプにあるサンプルコードを用いてExcelを操作してください.
  \mybto

\end{frame}

\end{document}

