\def\MyCourse{データサイエンスコース}
\def\MySubject{R入門}
\def\MySemester{春学期}

\newcommand{\R}{\textbf{R}}
\newcommand{\RStudio}{\textbf{RStudio}}
\newcommand{\Excel}{\textbf{Excel}}
\newcommand{\cs}[1]{\textcolor{blue}{\texttt{#1}}} % Console prompt >


\subsection{スカラの作成}

\myffr

  \mybfr{手順}
    オブジェクト名の後に,代入(付置)記号「<-」と値を入力する.
  \mybto
・「<-」の代わりに「=」も使用できる (若干意味が異なる).\\
・オブジェクト名は,大文字と小文字は区別される.
  %・ls()で作成したオブジェクトの名前を表示できる.\\
  %・rm('x')でオブジェクトxを削除できる. rm(list=ls())で全消去.\\

  \myee
  {コンソール1}
  {
    \cs{x <- 1}\\
    \cs{kw.pv <- 3.1}
  }
  {コンソール2}
  {
    \cs{ls()} \mycheck{オブジェクト名表示}\\
    \cs{rm(list=ls())} \mycheck{全消去}
  }

%  \R ではオブジェクト名称の大文字と小文字は区別される.
%  漢字名称も使用可能だが,通常はローマ字小文字で,
%  判別可能な略語を用い,ofの意味で「.」や「\_」
%  を利用し単語を結合すると分かりやすくなる.
%  【例】lat.jp (日本の緯度),kw.pv(PV発電量)

  \mybfr{演習}
    「Alt~+~-」で代入記号(<-)を20回入力してください.\\
    オブジェクト名を表示,オブジェクトを削除してください.
  \mybto

\end{frame}

\end{document}

