\def\MyCourse{データサイエンスコース}
\def\MySubject{R入門}
\def\MySemester{春学期}

\newcommand{\R}{\textbf{R}}
\newcommand{\RStudio}{\textbf{RStudio}}
\newcommand{\Excel}{\textbf{Excel}}
\newcommand{\cs}[1]{\textcolor{blue}{\texttt{#1}}} % Console prompt >


\subsection{ベクトルの作成}
\myffr

\mybbb
{手順(方法1)}{結合関数「c」を用いて作成}
{手順(方法2)}{等差数列作成記号「:」を用いて作成}
{手順(方法3)}{等差数列作成関数「seq」を用いて作成}\\[1mm]

「rep」関数で同一値ベクトル作成も可能
rep(NA, 3) $\rightarrow$ NA NA NA\\

「?関数名」をコンソールに入力するとヘルプが表示される.


\myeee
{コンソール1}{\cs{v <- c(1,6,3)}\\   \relax [1] 1 6 3}
{コンソール2}{\cs{v <- 1:3}\\        \relax [1] 1 2 3}
{コンソール3}{\cs{v <- seq(1,6,2)}\\ \relax [1] 1 3 5}\\[1mm]


\myb{演習}{
 次のベクトルを作成してください.\\
 「3 2 1」,「3 6 9」,「4 2 0」,「1.5 2.5 3.5」,「1 2 3 1 2 3」
}

\end{frame}

\include{def/footer}
