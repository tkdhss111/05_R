\def\MyCourse{データサイエンスコース}
\def\MySubject{R入門}
\def\MySemester{春学期}

\newcommand{\R}{\textbf{R}}
\newcommand{\RStudio}{\textbf{RStudio}}
\newcommand{\Excel}{\textbf{Excel}}
\newcommand{\cs}[1]{\textcolor{blue}{\texttt{#1}}} % Console prompt >


\subsection{データフレームの作成}

\myffr

  \mybfr{手順}
    テーブル作成関数「data.frame」を用いて作成する.
  \mybto

  \myefr{コンソール}
    \cs{d <- data.frame(name = c('panda', 'lion'),\\\hfill age  = c(5, 7), male = c(T, F))}\\
    \cs{d}\\
       ~~~name age  male\\
     1 panda   5  TRUE\\
     2 ~~~lion   7 FALSE\\
  \myeto

  \mybfr{演習}
    漢字,数値,論理値のカラムを持つ,
    データフレームを作成してください(内容自由).
  \mybto

\end{frame}

\subsection{データフレームの操作1}

\myffr

  \vspace{-4mm}

  \mybfr{手順}
    アクセスしたいコラム(列)やレコード(行)の
    インデック番号を入力する.
    負の番号を入れると,そのコラムが除かれる.
  \mybto

  \myeee
  {コンソール1}
  {
    \cs{ d[, 1] }\\ \relax
    [1] "panda" "lion" 
  }
  {コンソール2}
  {
    \cs{ d[, c(1, 3)] }\\
    ~~name  male\\
    1 panda  TRUE\\
    2 ~~~lion FALSE
  }
  {コンソール3}
  {
    \cs{ d[, -1] }\\
    ~~age  male\\
    1   5  TRUE\\
    2   7 FALSE
  }

  \mybfr{演習}
    レコード(行)にもインデックス番号でアクセスし,値の表示や代入を行ってください.
  \mybto

\end{frame}

\subsection{データフレームの操作2}

\myffr

  \vspace{-2mm}

  \mybfr{手順}
    オブジェクト名のあとにアクセスしたいコラム(列)名を$で結びつける.
    または,コラム(列)名をリテラルで囲み記入する.\\
    データフレームdのカラム: d\$カラム名 or d[, 'カラム名']
  \mybto

  \vspace{-2mm}

  \myee
    {コンソール1}
    {
      \cs{ d\$age }\\ \relax
      [1] 5 7
    }
    {コンソール2}
    {
      \cs{ d[, c('name', 'age')] }\\
      ~~~name age\\
        1 panda   5\\
        2 ~~~lion   7
    }

  \mybfr{演習}
  レコード(行)にもレコード名でアクセスし,値の表示や代入を行ってください.
  \mybto

  カラム名,レコード名はrownames(d), colnames(d)でアクセス可能
\end{frame}

\subsection{データフレームの操作3}

\myffr

  \mybfr{手順}
    アクセスしたいコラム(列)やレコード(行)に論理値を入力する.
    T(TRUE)の論理値箇所のデータが抽出される.
  \mybto

  \myee
    {コンソール1}
    {
      \cs{ d[, c(T, F, T)] }\\
          ~~~name  male\\
        1 panda  TRUE\\
        2 ~~~lion FALSE
    }
    {コンソール2}
    {
      \cs{ d[d\$age > 6, ] }\\
       ~~name age  male\\
       2 ~~lion ~~7 FALSE
     }

  \mybfr{演習}
    レコード(行)に論理値ベクトルでアクセスし,値の表示や代入を行ってください.
  \mybto

\end{frame}

\end{document}

