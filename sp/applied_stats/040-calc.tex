\def\MyCourse{データサイエンスコース}
\def\MySubject{R入門}
\def\MySemester{春学期}

\newcommand{\R}{\textbf{R}}
\newcommand{\RStudio}{\textbf{RStudio}}
\newcommand{\Excel}{\textbf{Excel}}
\newcommand{\cs}[1]{\textcolor{blue}{\texttt{#1}}} % Console prompt >


\subsection{四則}

\myffr

  \mybfr{手順}
    和「+」,減「-」,積「*」,除「/」,乗「\^\ 」,
    剰余「\%\%」,剰商「\%/\%」の算法記号を使って演算する.
    要素ごとの演算となる.
  \mybto
  この他,行列演算用の積「\%*\%」,転置「t()」,逆行列「solve()」などもある.

  \myeee
  {コンソール1}
  {
    \cs{x <- 1:3; y <- 1:3}\\
    \cs{x + y}\\ \relax
    [1] 2 4 6
  }
  {コンソール2}
  {
    \cs{x <- 1:3; y <- 1:3}\\
    \cs{x * y}\\ \relax
    [1] 1 4 9
  }
  {コンソール3}
  {
    \cs{x <- 9; y <- 2}\\
    \cs{x \%/\% y}\\ \relax
    [1] 4
  }

  \mybfr{演習}
    上記,すべての演算記号を使って,計算してください(内容自由).
  \mybto
  
\end{frame}

\subsection{組込関数1}

\myffr

  \mybfr{手順}
  平均値「mean」,中央値「median」,最大「max」,最小「min」\\
  範囲「range」,平方根「sqrt」,絶対値「abs」,丸め「round」など
  \mybto

  \myeee
  {コンソール1}
  {
    \cs{x <- 1:3}\\
    \cs{mean(x)}\\ \relax
    [1] 2
  }
  {コンソール2}
  {
    \cs{x <- 1:3}\\
    \cs{range(x)}\\ \relax
    [1] 1 3
  }
  {コンソール3}
  {
    \cs{x <- 3.14}\\
    \cs{round(x, 1)}\\ \relax
    [1] 3.1 
  }

  \mybfr{演習}
    上記,すべての組み込み関数を使って,計算してください(内容自由).
    また,その他の関数,定数(pi)は,どのようなものがあるかインターネットで検索してください.
  \mybto
  
\end{frame}

\subsection{組込関数2}

\myffr

  \mybfr{手順}
  平均値「mean」,中央値「median」,最大「max」,最小「min」\\
  範囲「range」,平方根「sqrt」,絶対値「abs」,丸め「round」など
  \mybto

  \myeee
  {コンソール1}
  {
    \cs{x <- 1:3}\\
    \cs{mean(x)}\\ \relax
    [1] 2
  }
  {コンソール2}
  {
    \cs{x <- 1:3}\\
    \cs{range(x)}\\ \relax
    [1] 1 3
  }
  {コンソール3}
  {
    \cs{x <- 3.14}\\
    \cs{round(x, 1)}\\ \relax
    [1] 3.1 
  }

  \mybfr{演習}
    上記,すべての組み込み関数を使って,計算してください(内容自由).
    また,その他の関数,定数(pi)は,どのようなものがあるかインターネットで検索してください.
  \mybto
  
\end{frame}

\subsection{自作関数1}
\myffr

  \mybfr{手順}
    関数オブジェクト名 <- function (引数1,引数2,...) 関数式\\
    の形式で関数を作成する.引数は値渡しとなる.
  \mybto

  \myefr{コンソール}
    \cs{get.mbe <- function(yhat, y) mean(yhat - y)}\\
    \cs{mbe <- get.mbe(yhat \textcolor{red}{=} 1:3, y \textcolor{red}{=} 4:6)}\\\relax
    [1] -3
  \myeto

  \mybfr{演習}
    RMSE(平均2乗誤差平方)を求める関数を作成してください.
    \vspace{-2mm}
    \[
    \mathrm{RMSE} = \sqrt{\frac{1}{N} \sum_{i = 1}^{N} (\hat{y}_i-y_i)^2}
    \hspace{5mm} \mbox{平均:mean(),平方根:sqrt()}
    \]
  \mybto
  
\end{frame}

\subsection{自作関数2}

\myffr

  \mybfr{手順}
    複数行の関数を作成するときは,関数のスコープを示す{}(中括弧)や変数名を指定して出力するreturn関数を用いる.
  \mybto 

  \myee
  {コンソール1}
  {
    \cs{f <- function(x) \{ \mycheck{}\\
        ~~~~~y~<- 1 + x + x \^ \ 2\\
        ~~~~~z~<- log(y)\\
        ~~~~~return(z) \mycheck{}\\
        ~~~\} \mycheck{}
    }
  }
  {コンソール2}
  {
    \cs{f(2)}\\\relax
    [1] 1.94591
  }\\[2mm]

  \mycheck{ return関数では,最後のオブジェクトを返す場合は記述しなくてもよい.}

  \mybfr{演習}
    複数行の関数を作成してください.
  \mybto
  
\end{frame}

\end{document}

